%%%%%%%%%%%%%%%%%%%%%%%%%%%%%%%%%%%%%%%%%
% Short Sectioned Assignment LaTeX Template Version 1.0 (5/5/12)
% This template has been downloaded from: http://www.LaTeXTemplates.com
% Original author:  Frits Wenneker (http://www.howtotex.com)
% License: CC BY-NC-SA 3.0 (http://creativecommons.org/licenses/by-nc-sa/3.0/)
%%%%%%%%%%%%%%%%%%%%%%%%%%%%%%%%%%%%%%%%%

%----------------------------------------------------------------------------------------
%	PACKAGES AND OTHER DOCUMENT CONFIGURATIONS
%----------------------------------------------------------------------------------------

\documentclass[paper=a4, fontsize=11pt]{scrartcl} % A4 paper and 11pt font size

% ---- Entrada y salida de texto -----

\usepackage[T1]{fontenc} % Use 8-bit encoding that has 256 glyphs
\usepackage[utf8]{inputenc}

% ---- Idioma --------

\usepackage[spanish, es-tabla]{babel} % Selecciona el español para palabras introducidas automáticamente, p.ej. "septiembre" en la fecha y especifica que se use la palabra Tabla en vez de Cuadro

% ---- Otros paquetes ----

\usepackage{amsmath,amsfonts,amsthm} % Math packages
\usepackage{graphics,graphicx, floatrow} %para incluir imágenes y notas en las imágenes
\usepackage{graphics,graphicx, float} %para incluir imágenes y colocarlas
\usepackage{hyperref} % url in references

% Para hacer tablas comlejas
\usepackage{multirow}
\usepackage{threeparttable}

\usepackage{fancyhdr} % Custom headers and footers
\pagestyle{fancyplain} % Makes all pages in the document conform to the custom headers and footers
\fancyhead{} % No page header - if you want one, create it in the same way as the footers below
\fancyfoot[L]{} % Empty left footer
\fancyfoot[C]{} % Empty center footer
\fancyfoot[R]{\thepage} % Page numbering for right footer
\renewcommand{\headrulewidth}{0pt} % Remove header underlines
\renewcommand{\footrulewidth}{0pt} % Remove footer underlines
\setlength{\headheight}{13.6pt} % Customize the height of the header

\numberwithin{equation}{section} % Number equations within sections (i.e. 1.1, 1.2, 2.1, 2.2 instead of 1, 2, 3, 4)
\numberwithin{figure}{section} % Number figures within sections (i.e. 1.1, 1.2, 2.1, 2.2 instead of 1, 2, 3, 4)
\numberwithin{table}{section} % Number tables within sections (i.e. 1.1, 1.2, 2.1, 2.2 instead of 1, 2, 3, 4)

\setlength\parindent{0pt} % Removes all indentation from paragraphs - comment this line for an assignment with lots of text

\newcommand{\horrule}[1]{\rule{\linewidth}{#1}} % Create horizontal rule command with 1 argument of height

\usepackage{textcomp}

%----------------------------------------------------------------------------------------
%	DATOS
%----------------------------------------------------------------------------------------

\newcommand{\myName}{Francisco Javier Bolívar Lupiáñez}
\newcommand{\myDegree}{Máster en Ingeniería Informática}
\newcommand{\myFaculty}{E. T. S. de Ingenierías Informática y de Telecomunicación}
\newcommand{\myDepartment}{Ciencias de la Computación e Inteligencia Artificial}
\newcommand{\myUniversity}{\protect{Universidad de Granada}}
\newcommand{\myLocation}{Granada}
\newcommand{\myTime}{\today}
\newcommand{\myTitle}{Práctica 2}
\newcommand{\mySubtitle}{Algoritmos evolutivos. Resolución del problema NP QAP}
\newcommand{\mySubject}{Inteligencia Computacional}
\newcommand{\myYear}{2016-2017}

%----------------------------------------------------------------------------------------
%	PORTADA
%----------------------------------------------------------------------------------------


\title{	
	\normalfont \normalsize 
	\textsc{\textbf{\mySubject \space (\myYear)} \\ \myDepartment} \\[20pt] % Your university, school and/or department name(s)
	\textsc{\myDegree \\[10pt] \myFaculty \\ \myUniversity} \\[25pt]
	\horrule{0.5pt} \\[0.4cm] % Thin top horizontal rule
	\huge \myTitle: \mySubtitle \\ % The assignment title
	\horrule{2pt} \\[0.5cm] % Thick bottom horizontal rule
	\normalfont \normalsize
}

\author{\myName} % Nombre y apellidos

\date{\myTime} % Incluye la fecha actual
%----------------------------------------------------------------------------------------
%	INDICE
%----------------------------------------------------------------------------------------

\begin{document}
	
\setcounter{page}{0}

\maketitle % Muestra el Título
\thispagestyle{empty}

\newpage %inserta un salto de página

\tableofcontents % para generar el índice de contenidos

%\listoffigures

\newpage

%----------------------------------------------------------------------------------------
%	DOCUMENTO
%----------------------------------------------------------------------------------------

\section{Introducción}

Durante esta práctica se han implementado tres variantes de un algoritmo evolutivo (algoritmo genético estándar, variante balwiniana y lamarckiana) para resolver el problema NP de la asignación cuadrática o QAP (Quadratic Assignment Problem).
\\ \\
El problema consiste en designar una serie de localizaciones a instalaciones, dadas las distancias y flujo de materiales entre cada una de ellas, para minimizar el coste de transporte de materiales entre instalaciones.
\\ \\
La función de coste es:
\[ \sum_{i,j} w(i,j) \times d(p(i),p(j)) \]
Siendo:
\begin{itemize}
	\item $ w(i,j) $ el peso asociado al flujo de materiales transportados desde la instalación i a la instalación j
	\item $ d(i,j) $ la distancia de la localización i a la j
	\item $ p(i) $ la instalación i en una posible solución del problema
\end{itemize}
Los casos de prueba han sido obtenidos de la biblioteca QAPLIB \footnote{\url{http://www.seas.upenn.edu/qaplib/}}. Los ficheros tienen el siguiente formato:
\\ \\
\texttt{n}
\\
\texttt{D}
\\
\texttt{W}
\\ \\
Donde \textit{n} es el tamaño del problema, \textit{D} es la matriz de tamaño $ n \times n $ de distancias y \textit{W} es la matriz de tamaño $ n \times n $ de flujos de material.
\\ \\
El objetivo de la práctica es intentar obtener el mejor resultado posible sobre el conjunto de datos de prueba \texttt{tai256c}. Actualmente la mejor solución obtenida con un algoritmo evolutivo para este problema es la de una permutación con un coste de 44759294.

\section{Implementación}

En esta sección se tratarán los temas de implementación de las distintas variantes.

\subsection{Algoritmo genético estándar}

En primer lugar se realizó un algoritmo genético estándar sin optimización local. Para ello había que tomar varias decisiones en los distintos operadores.

\subsubsection{Mecanismo de selección}

Lala

\subsubsection{Mecanismo de reemplazo}

Lala

\subsubsection{Operador de cruce}

Lala

\subsubsection{Operador de mutación}

Lala

\subsection{Variante \textit{baldwiniana}}

Lala

\subsubsection{Optimización local}

Lala

\subsection{Variante \textit{lamarckiana}}

Lala

\section{Resultados}

Lala

\section{Conclusiones}

Lala


%----------------------------------------------------------------------------------------
%	REFERENCIAS
%----------------------------------------------------------------------------------------

\newpage

\bibliography{referencias} %archivo referencias.bib que contiene las entradas 
\bibliographystyle{plain} % hay varias formas de citar

\end{document}